\pagebreak

\renewcommand{\contentsname}{Sommaire} % Dans le corps du document,avant la commande \tableofcontents.

\setcounter{tocdepth}{5}

\addcontentsline{toc}{section}{Sommaire}

\tableofcontents

\pagebreak


%\addcontentsline{toc}{section}{Références}

	% Pour compiler le document, vous devrez exécuter 3 commandes en séquence:

	%	pdflatex pour créer un fichier auxiliaire qui indique à biber quelles sources sont nécessaires
	%  biber pour créer un fichier auxiliaire avec toutes les sources utilisables par pdflatex
	%  pdflatex pour inclure le fichier auxiliaire et créer le PDF 



\printbibliography
\textbf{Cette liste de références contient également des lectures suggérées pour approfondir le sujet.}\\

	
	
\vspace{1cm}	
\textbf{\textit{Ce document a été écrit avec LaTeX}.}\\
	
	
	
	
	
	@Jean-Pierre Duchesneau
	\vspace{1cm}
	
	\shadowbox{\parbox{15cm}
		{Cette création est mise à disposition selon le contrat Attribution-Partage dans les mêmes conditions 4.0 International de Creative Commons. En vertu de ce contrat, vous êtes libre de :
			\begin{itemize}
				\item \textbf{partager} -- reproduire, distribuer et communiquer l’œuvre ;
				\item \textbf{remixer} --adapter l’œuvre ;
				\item  utiliser cette œuvre à des fins commerciales.
			\end{itemize}
			Selons les conditions suivantes :\newline
			\textbf{Attribution} — Vous devez créditer l’œuvre, intégrer un lien vers le
			contrat et indiquer si des modifications ont été effectuées à l’œuvre.
			Vous devez indiquer ces informations par tous les moyens posSibles, 
			mais vous ne pouvez suggérer que l’Offrant vous soutient
			ou soutient la façon dont vous avez utilisé son œuvre.
			
			\textbf{Partage dans les mêmes conditions} — Dans le cas où vous modifiez, 
			transformez ou créez à partir du matériel composant l’œuvre
			originale, vous devez diffuser l’œuvre modifiée dans les même
			conditions, c’est à dire avec le même contrat avec lequel l’œuvre
			originale a été diffusée.
	}}
	
	\vspace{2cm}